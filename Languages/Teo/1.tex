\mytitle{\Large Gramática de la Fantasía - Rodari}\\[0.1cm]

\begin{itemize}
	\item {\bf Juego de la china} Partiendo de una palabra, explorarla. Obtener cadenas de palabras por cercanía fónica. Indagar si tiene polisemia. Crear acrósticos. {\it Una analogía verbal puede sucitar una metáfora}.
	\item {\bf El prefijo arbitrario} Se modifica una palabra agregandole un prefijo cualquiera.
	\item {\bf El error creativo} Aprovechar el error ortográfico como una creación autónoma. Los niños suelen saltar de una palabra desconocida a una conocida por familiaridad fónica, y esto para asimilar una realidad desconocida.
	\item {\bf El binomio fantástico} {\it Aquí las palabras no son tomadas en su significado cotidiano, sino liberadas de las cadenas verbales de las que forman parte cotidianamente}. Enlazarlas con una preposición y los correspondientes artículos. Crear contexto para dos sustantivos.
	\item {\bf La trama fantástica} Se eligen al azar un sujeto y un predicado. Su unión proporcionará la hipótesis fantástica sobre la cual trabajar. ¿Qué pasaría si...
	\item {\bf La sintaxis fantástica} Se elige una serie de preguntas al azar (¿Quién era?, ¿dónde estaba?, ¿qué hacía?, ¿qué dijo?, etc). El movimiento del {\it nonsense} al sentido.
	\item {\bf Tratamiento de un verso} Elegir algún verso y reescribirlo deformándolo, en la búsqueda de un tema fantástico. El ejercicio es para adiestrar a la imaginación de apartarse de los caminos demasiado comunes de significado.
	\item {\bf Falsa adivinanza} Es aquella que contiene ya de uno u otro modo su respuesta.

	\item {\bf La doble caída} Del rito a lo laico, de lo laico al juego ({\it Máscara ritual, traje teatral, marioneta}).
	\item {\bf Decodificar el lado cómico} Interpretar la historia
	 favoreciendo la resolución cómica frente a la patética.
	\item {\bf Análisis fantástico} Sobre un personaje de cuento, consistirá en descmponerlo en sus factores primarios, con el fin de rastrear los elementos para la construcción de nuevos {\it binomios fantásticos}. Esto es, inventar nuevas historias en torno al personaje. El personaje elegido puede ser conocido ({\it Pulgarcito}), arquetípico ({\it el vaquero, el brujo}) o producto de un binomio ({\it el hombre de vidrio}). Sus aventuras pueden deducirse de sus características.
	\item {{\bf Construcción de un Limerick}
		\begin{enumerate}
			\item Elección del {\it protagonista}
			\item Indicación de una {\it cualidad}, expresada con una acción
			\item Realización del {\it predicado} en el 3º y 4º verso
			\item Elección del {\it epíteto} final
			\item Los versos 1º, 2º, 5º y 3º, 4º riman entre sí
		\end{enumerate}}
	\item {{\bf Construcción de una adivinanza}
		\begin{enumerate}
			\item {\it Extrañamiento}, definir al objeto como visto por primera vez
			\item {\it Asociación}, de la definición crear aperturas a otros significados a partir de imágenes por analogía o comparación
			\item {\it Metáfora}, crear una nueva definición metafórica
		\end{enumerate}}
	\item {{\bf Tratamiento de cuentos}
		\begin{enumerate}
			\item {\it Binomio}, usar al cuanto como el primer término del binomio y a una palabra inesperada como el segundo
			\item {\it ¿Qué ocurre después?}, partiendo de las reglas internas del cuento y mediante una nueva palabra, crear una continuación
			\item {\it Los cuentos al revés}, invertir los valores de los personajes, trama
			\item {\it Ensalada de cuentos}, usar a un cuento como un término del binomio y a otro como el otro
			\item {\it Imitando cuentos}, usar al cuento como un complejo sistema de coordenadas fantásticas.
			\item {\it Transformación de un tema}, por reducción, amplificación, sustitución e intensificación
		\end{enumerate}}
	\item {{\bf Principios de Propp}
		\begin{enumerate}
			\item Los elementos constantes son las funciones de los personajes
			\item El número de funciones que incluye el cuento es limitado
			\item La sucesión de funciones es siempre idéntica
		\end{enumerate}
		Funciones: alejamiento, prohibición, transgreción, interrogatorio, información, engaño, complicidad, fechoría, mediación, principio de la acción contraria, partida, primera función del donante, reacción del héroe, recepción del objeto mágico, desplazamiento, combate, marca, victoria, reparación, la vuelta, persecución, socorro, llegada de incógnito, pretensiones engañosas, tarea difícil, tarea cumplida, reconocimiento, descubrimiento, transfiguración, castigo, matrimonio.}
\end{itemize}